% Options for packages loaded elsewhere
\PassOptionsToPackage{unicode}{hyperref}
\PassOptionsToPackage{hyphens}{url}
\PassOptionsToPackage{dvipsnames,svgnames,x11names}{xcolor}
%
\documentclass[
  11pt,
]{article}

\usepackage{amsmath,amssymb}
\usepackage{iftex}
\ifPDFTeX
  \usepackage[T1]{fontenc}
  \usepackage[utf8]{inputenc}
  \usepackage{textcomp} % provide euro and other symbols
\else % if luatex or xetex
  \usepackage{unicode-math}
  \defaultfontfeatures{Scale=MatchLowercase}
  \defaultfontfeatures[\rmfamily]{Ligatures=TeX,Scale=1}
\fi
\usepackage{lmodern}
\ifPDFTeX\else  
    % xetex/luatex font selection
\fi
% Use upquote if available, for straight quotes in verbatim environments
\IfFileExists{upquote.sty}{\usepackage{upquote}}{}
\IfFileExists{microtype.sty}{% use microtype if available
  \usepackage[]{microtype}
  \UseMicrotypeSet[protrusion]{basicmath} % disable protrusion for tt fonts
}{}
\makeatletter
\@ifundefined{KOMAClassName}{% if non-KOMA class
  \IfFileExists{parskip.sty}{%
    \usepackage{parskip}
  }{% else
    \setlength{\parindent}{0pt}
    \setlength{\parskip}{6pt plus 2pt minus 1pt}}
}{% if KOMA class
  \KOMAoptions{parskip=half}}
\makeatother
\usepackage{xcolor}
\usepackage[margin=1in]{geometry}
\setlength{\emergencystretch}{3em} % prevent overfull lines
\setcounter{secnumdepth}{-\maxdimen} % remove section numbering
% Make \paragraph and \subparagraph free-standing
\makeatletter
\ifx\paragraph\undefined\else
  \let\oldparagraph\paragraph
  \renewcommand{\paragraph}{
    \@ifstar
      \xxxParagraphStar
      \xxxParagraphNoStar
  }
  \newcommand{\xxxParagraphStar}[1]{\oldparagraph*{#1}\mbox{}}
  \newcommand{\xxxParagraphNoStar}[1]{\oldparagraph{#1}\mbox{}}
\fi
\ifx\subparagraph\undefined\else
  \let\oldsubparagraph\subparagraph
  \renewcommand{\subparagraph}{
    \@ifstar
      \xxxSubParagraphStar
      \xxxSubParagraphNoStar
  }
  \newcommand{\xxxSubParagraphStar}[1]{\oldsubparagraph*{#1}\mbox{}}
  \newcommand{\xxxSubParagraphNoStar}[1]{\oldsubparagraph{#1}\mbox{}}
\fi
\makeatother


\providecommand{\tightlist}{%
  \setlength{\itemsep}{0pt}\setlength{\parskip}{0pt}}\usepackage{longtable,booktabs,array}
\usepackage{calc} % for calculating minipage widths
% Correct order of tables after \paragraph or \subparagraph
\usepackage{etoolbox}
\makeatletter
\patchcmd\longtable{\par}{\if@noskipsec\mbox{}\fi\par}{}{}
\makeatother
% Allow footnotes in longtable head/foot
\IfFileExists{footnotehyper.sty}{\usepackage{footnotehyper}}{\usepackage{footnote}}
\makesavenoteenv{longtable}
\usepackage{graphicx}
\makeatletter
\newsavebox\pandoc@box
\newcommand*\pandocbounded[1]{% scales image to fit in text height/width
  \sbox\pandoc@box{#1}%
  \Gscale@div\@tempa{\textheight}{\dimexpr\ht\pandoc@box+\dp\pandoc@box\relax}%
  \Gscale@div\@tempb{\linewidth}{\wd\pandoc@box}%
  \ifdim\@tempb\p@<\@tempa\p@\let\@tempa\@tempb\fi% select the smaller of both
  \ifdim\@tempa\p@<\p@\scalebox{\@tempa}{\usebox\pandoc@box}%
  \else\usebox{\pandoc@box}%
  \fi%
}
% Set default figure placement to htbp
\def\fps@figure{htbp}
\makeatother
% definitions for citeproc citations
\NewDocumentCommand\citeproctext{}{}
\NewDocumentCommand\citeproc{mm}{%
  \begingroup\def\citeproctext{#2}\cite{#1}\endgroup}
\makeatletter
 % allow citations to break across lines
 \let\@cite@ofmt\@firstofone
 % avoid brackets around text for \cite:
 \def\@biblabel#1{}
 \def\@cite#1#2{{#1\if@tempswa , #2\fi}}
\makeatother
\newlength{\cslhangindent}
\setlength{\cslhangindent}{1.5em}
\newlength{\csllabelwidth}
\setlength{\csllabelwidth}{3em}
\newenvironment{CSLReferences}[2] % #1 hanging-indent, #2 entry-spacing
 {\begin{list}{}{%
  \setlength{\itemindent}{0pt}
  \setlength{\leftmargin}{0pt}
  \setlength{\parsep}{0pt}
  % turn on hanging indent if param 1 is 1
  \ifodd #1
   \setlength{\leftmargin}{\cslhangindent}
   \setlength{\itemindent}{-1\cslhangindent}
  \fi
  % set entry spacing
  \setlength{\itemsep}{#2\baselineskip}}}
 {\end{list}}
\usepackage{calc}
\newcommand{\CSLBlock}[1]{\hfill\break\parbox[t]{\linewidth}{\strut\ignorespaces#1\strut}}
\newcommand{\CSLLeftMargin}[1]{\parbox[t]{\csllabelwidth}{\strut#1\strut}}
\newcommand{\CSLRightInline}[1]{\parbox[t]{\linewidth - \csllabelwidth}{\strut#1\strut}}
\newcommand{\CSLIndent}[1]{\hspace{\cslhangindent}#1}

\usepackage{setspace}
\doublespacing
\makeatletter
\@ifpackageloaded{caption}{}{\usepackage{caption}}
\AtBeginDocument{%
\ifdefined\contentsname
  \renewcommand*\contentsname{Table of contents}
\else
  \newcommand\contentsname{Table of contents}
\fi
\ifdefined\listfigurename
  \renewcommand*\listfigurename{List of Figures}
\else
  \newcommand\listfigurename{List of Figures}
\fi
\ifdefined\listtablename
  \renewcommand*\listtablename{List of Tables}
\else
  \newcommand\listtablename{List of Tables}
\fi
\ifdefined\figurename
  \renewcommand*\figurename{Figure}
\else
  \newcommand\figurename{Figure}
\fi
\ifdefined\tablename
  \renewcommand*\tablename{Table}
\else
  \newcommand\tablename{Table}
\fi
}
\@ifpackageloaded{float}{}{\usepackage{float}}
\floatstyle{ruled}
\@ifundefined{c@chapter}{\newfloat{codelisting}{h}{lop}}{\newfloat{codelisting}{h}{lop}[chapter]}
\floatname{codelisting}{Listing}
\newcommand*\listoflistings{\listof{codelisting}{List of Listings}}
\makeatother
\makeatletter
\makeatother
\makeatletter
\@ifpackageloaded{caption}{}{\usepackage{caption}}
\@ifpackageloaded{subcaption}{}{\usepackage{subcaption}}
\makeatother

\usepackage{bookmark}

\IfFileExists{xurl.sty}{\usepackage{xurl}}{} % add URL line breaks if available
\urlstyle{same} % disable monospaced font for URLs
\hypersetup{
  pdftitle={Diet Sodas: A Deep Dive into the Causal Effect of Artificial Sweeteners on Obesity},
  colorlinks=true,
  linkcolor={blue},
  filecolor={Maroon},
  citecolor={blue},
  urlcolor={blue},
  pdfcreator={LaTeX via pandoc}}


\title{Diet Sodas: A Deep Dive into the Causal Effect of Artificial
Sweeteners on Obesity}
\author{Chase Clemence}
\date{8 Aug 2025}

\begin{document}
\maketitle

\renewcommand*\contentsname{Table of contents}
{
\hypersetup{linkcolor=}
\setcounter{tocdepth}{2}
\tableofcontents
}

\subsection{Abstract}\label{abstract}

\subsection{Introduction}\label{introduction}

Obesity is a growing public health issue in the United States, with the
epidemic continuing to worsen each year. According to the Centers for
Disease Control and Prevention (CDC), the majority of U.S. states
reported an obesity prevalence of over 30\%, indicating that nearly
one-third of Americans have a Body Mass Index (BMI) of 30 or higher
(Centers for Disease Control and Prevention (CDC) 2024). Although
obesity is a nationwide concern, it is especially concentrated in the
Midwest and South. In Texas, for instance, approximately 52\% of males
aged 15--24 were classified as overweight or obese in 2021. Even more
striking, nearly 80\% of men over the age of 25 in North Dakota were
considered at least overweight. Future projections for the American
obesity epidemic are more troubling. By the year 2050, it is estimated
that one-fifth of children (ages 5--15), one-third of adolescents (ages
15--24), and two-thirds of adults (ages 25 and older) will be classified
as obese (Institute for Health Metrics and Evaluation (IHME) 2024).
While BMI is not a perfect measure of health---often misclassifying
highly muscular individuals and senior citizens---it remains a broadly
accepted indicator of population-level health trends. Although genetics
may predispose some individuals to higher weight, obesity in the U.S. is
primarily driven by the foods and beverages that Americans regularly
consume.

The fast-paced American lifestyle encourages reliance on processed and
ultra-processed foods, which are made affordable and accessible by the
fast-food and packaged food industries. According to Dr.~Leigh
Frame---program director for the Integrative Medicine Programs and a
clinical researcher at George Washington University---this dietary
pattern is closely tied to weight gain, elevated fasting blood glucose,
microbiome disruptions, and other health complications (Laster and Frame
2019). Regarding weight gain specifically, Frame and her colleagues cite
a controlled inpatient study in which 20 adults were randomly assigned
to follow either an ultra-processed or an unprocessed diet for two
weeks. These diets were nutritionally matched in perceived calories,
sugar, fat, fiber, and macronutrient content, yet participants on the
ultra-processed diet consumed significantly more energy overall,
particularly from carbohydrates and fats. This led to an average weight
gain of 0.9 kg (p = 0.009), while those on the unprocessed diet lost a
similar amount (p = 0.007) (Hall et al. 2019). These findings underscore
the impact of food processing on energy intake and weight. While diet
sodas are often marketed as a healthier alternative to sugary drinks,
they remain ultra-processed and may pose similar risks.

Beverage companies often tout diet sodas as a zero-calorie,
health-conscious choice. For example, PepsiCo's finance chief in 2023
claimed that aspartame-sweetened drinks ``obviously have the benefit of
being zero calorie'' (Rajesh 2023). However, a careful look at the
science reveals more nuance. In a 2022 meta-analysis of randomized
trials, McGlynn \emph{et al.} found that substituting low- or no-calorie
drinks for sugar-sweetened sodas led to only a small average weight loss
(about --1.06 kg) (McGlynn et al. 2022). By contrast, large cohort
studies consistently link frequent diet-soda consumption with higher
risks of weight gain, obesity, diabetes, and cardiovascular disease
(McGlynn et al. 2022). Likewise, Toews \emph{et al.} (2019) reported
that evidence from clinical and observational studies is generally low
quality and shows essentially no long-term benefit of artificial
sweeteners on weight (Toews et al. 2019). In fact, some data suggest
diet-drink users may even compensate by eating more or experience
altered metabolism---for example, sweeteners can disrupt normal appetite
signaling and gut microbiota in ways that promote weight gain (McGlynn
et al. 2022). Despite industry claims to the contrary (Rajesh 2023),
scientific reviews indicate that drinking diet sodas is not clearly
protective against obesity and may be associated with lasting metabolic
harms (McGlynn et al. 2022; Toews et al. 2019).

This paper explores the causal impact of diet soda consumption on weight
gain, addressing a public health issue that remains largely shrouded in
uncertainty. While many consumers believe they are making a healthier
choice by switching to diet alternatives, the evidence presented here
suggests that diet soda may cause more harm than good in the long term.

\subsection{Data}\label{data}

All data used in this analysis were obtained from the National Health
and Nutrition Examination Survey (NHANES), a nationally representative
program that collects detailed information on the dietary habits and
nutrient intake of U.S. residents. NHANES employs a rigorous
methodology, compiling a wide range of data including demographic
information, dietary intake, physical examination findings, laboratory
results, self-reported questionnaire responses, and more. This study
focuses on a subset of these variables---specifically demographics,
financial status, physical fitness indicators, self-perceived diet
quality, and weight history---as potential confounders.

To increase reliability, NHANES provides two non-consecutive 24-hour
dietary recall datasets for each participant, spaced approximately 30
days apart. Although NHANES is conducted annually using consistent data
collection protocols, the quality and completeness of post-pandemic data
have declined. For this reason, data from the 2013--2014 cycle were
selected for this analysis, though the methods described here are
generalizable to other NHANES cycles.

The data cleaning process began with familiarization with NHANES food
codes to accurately identify participants who reported consuming diet
soda. Once these specific codes were isolated, participants were
categorized accordingly. Relevant covariates were then merged using the
SEQN variable, a unique identifier assigned to each participant. This
integration enabled a structured and meaningful examination of
diet-related patterns and outcomes. The table below outlines the names
and definitions of specific variables used.

\begin{longtable}[]{@{}
  >{\raggedright\arraybackslash}p{(\linewidth - 2\tabcolsep) * \real{0.2778}}
  >{\raggedright\arraybackslash}p{(\linewidth - 2\tabcolsep) * \real{0.7222}}@{}}
\toprule\noalign{}
\begin{minipage}[b]{\linewidth}\raggedright
Variable Name
\end{minipage} & \begin{minipage}[b]{\linewidth}\raggedright
\textbf{Description}
\end{minipage} \\
\midrule\noalign{}
\endhead
\bottomrule\noalign{}
\endlastfoot
\textbf{SEQN} & Respondent sequence number \\
\textbf{SODADRINKER} & A boolean variable that identifies if a
respondent drinks diet soda \\
\textbf{BMXBMI} & Body Mass Index (kg/m**2) \\
\textbf{RIAGENDR} & Gender of the participant \\
\textbf{RIDRETH1} & Reported race and Hispanic origin information \\
\textbf{RIDAGEYR} & Age in years of the participant at the time of
screening \\
\textbf{DMDEDU2} & Education level (Adults 20+) \\
\textbf{DMDEDU3} & Education level (Children 6-19) \\
\textbf{IND235} & Compare outcomes of diet soda consumers
vs.~non-consumers using covariates \\
\textbf{PAQ610} & Number of days per week where vigorous activity is
involved \\
\textbf{PAQ625} & Number of days per week where moderate activity is
involved \\
\textbf{WHQ060} & Binary variable specifying if weight change is
intentional \\
\textbf{DBQ700} & Categorical variable specifying the healthiness of the
participant's diet \\
\end{longtable}

\subsection{Methods}\label{methods}

\subsection{Results}\label{results}

\subsection{Conclusions}\label{conclusions}

\subsection*{Future Work}\label{future-work}
\addcontentsline{toc}{subsection}{Future Work}

\phantomsection\label{refs}
\begin{CSLReferences}{1}{0}
\bibitem[\citeproctext]{ref-cdc_adult_obesity_maps}
Centers for Disease Control and Prevention (CDC). 2024. {``Adult Obesity
Prevalence Maps.''}
\url{https://www.cdc.gov/obesity/data-and-statistics/adult-obesity-prevalence-maps.html}.

\bibitem[\citeproctext]{ref-hall2019ultra}
Hall, Kevin D., Abraham Ayuketah, Robert Brychta, Hongyi Cai, Tyler
Cassimatis, Kyle Y. Chen, Seongwon T. Chung, et al. 2019.
{``Ultra-Processed Diets Cause Excess Calorie Intake and Weight Gain: An
Inpatient Randomized Controlled Trial of Ad Libitum Food Intake.''}
\emph{Cell Metabolism} 30 (1): 67--77.e3.
\url{https://doi.org/10.1016/j.cmet.2019.05.008}.

\bibitem[\citeproctext]{ref-ihme_lancet_obesity2050}
Institute for Health Metrics and Evaluation (IHME). 2024. {``Without
Immediate Action, Nearly 260\,Million People in the u.s. Predicted to Be
Overweight or Obese by 2050.''}
\url{https://www.healthdata.org/news-events/newsroom/news-releases/lancet-without-immediate-action-nearly-260-million-people-usa}.

\bibitem[\citeproctext]{ref-laster_frame_2019}
Laster, Janese, and Leigh A. Frame. 2019. {``Beyond the Calories---Is
the Problem in the Processing?''} \emph{Current Treatment Options in
Gastroenterology} 17 (4): 577--86.
\url{https://doi.org/10.1007/s11938-019-00246-1}.

\bibitem[\citeproctext]{ref-McGlynn2022}
McGlynn, N. D., T. A. Khan, L. Wang, R. Zhang, L. Chiavaroli, F.
Au-Yeung, J. J. Lee, et al. 2022. {``Association of Low- and No-Calorie
Sweetened Beverages as a Replacement for Sugar-Sweetened Beverages with
Body Weight and Cardiometabolic Risk: A Systematic Review and
Meta-Analysis.''} \emph{JAMA Network Open} 5 (3): e222092.
\url{https://doi.org/10.1001/jamanetworkopen.2022.2092}.

\bibitem[\citeproctext]{ref-Rajesh2023}
Rajesh, Ananya Mariam. 2023. {``PepsiCo Says It Has No Plans to Change
Its Portfolio as WHO Set to Warn on Aspartame Sweeteners.''}
\emph{Reuters}.
\url{https://www.reuters.com/business/retail-consumer/pepsico-says-no-plans-change-portfolio-who-set-warn-aspartame-sweeteners-2023-07-13/}.

\bibitem[\citeproctext]{ref-Toews2019}
Toews, I., S. Lohner, D. Küllenberg de Gaudry, H. Sommer, and J. J.
Meerpohl. 2019. {``Association Between Intake of Non-Sugar Sweeteners
and Health Outcomes: Systematic Review and Meta-Analyses of Randomized
and Non-Randomized Controlled Trials and Observational Studies.''}
\emph{BMJ} 364: k4718. \url{https://doi.org/10.1136/bmj.k4718}.

\end{CSLReferences}




\end{document}
